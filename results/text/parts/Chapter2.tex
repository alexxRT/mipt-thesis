\section{Обзор существующих решений}
\label{sec:Chapter2} \index{Chapter2}

\subsection{Введение в раздел}

В условиях стремительного развития технологий машинного обучения особую актуальность приобретает задача эффективного выполнения сложных моделей на разнообразных аппаратных платформах.
Одним из ключевых компонентов современной инфраструктуры стал промежуточный язык представления MLIR (Multi-Level Intermediate Representation), который служит унифицирующим звеном между высокоуровневыми описаниями моделей и низкоуровневым оптимизированным кодом.
Тем не менее, существующая реализация MLIR демонстрирует ограниченные возможности для интеграции данных о производительности, что существенно затрудняет процесс комплексного анализа и оптимизации промежуточного представления.
Целью данного обзора является систематический анализ современных подходов к профилированию машинного обучения, инструментам анализа и манипуляции MLIR, а также методам визуализации производительности. В рамках исследования рассматриваются профилировщики ML-моделей, инструменты анализа и модификации промежуточного представления, системы метаданных и подходы к визуализации результатов профилирования.
Особое внимание уделяется выявлению возможностей автоматического аннотирования MLIR на основе данных профиля выполнения.
Методология анализа базируется на следующих критериях оценки:

\begin{itemize}
\item \textbf{Функциональность}: степень охвата задач профилирования и аннотирования промежуточного представления.
\item \textbf{Архитектурная совместимость}: уровень интеграции с инфраструктурой MLIR/LLVM.
\item \textbf{Масштабируемость}: способность эффективно работать с моделями большого размера.
\item \textbf{Расширяемость}: возможность кастомизации и добавления новых метрик.
\item \textbf{Применимость}: соответствие целевой платформе (CPU) и экосистеме TensorFlow.
\end{itemize}

Анализ проводится по категориям инструментов, после чего осуществляется сравнительный анализ, направленный на выявление ключевых пробелов и перспектив интеграции различных решений.

\subsection{Профилировщики машинного обучения}

\subsubsection{TensorFlow Profiler}
TensorFlow Profiler \cite{tf_profiler} представляет собой комплексный инструмент для анализа производительности, интегрированный непосредственно в экосистему TensorFlow. Архитектура профилировщика основана на тесной интеграции с TensorFlow Runtime, что обеспечивает сбор метрик на уровне отдельных операций и ядер вычислений при поддержке различных вычислительных бэкендов, включая CPU, GPU и TPU. Выходные данные профилировщика формируются в формате trace events (JSON), содержащем структурированную информацию о времени выполнения, использовании памяти и утилизации вычислительных ресурсов.
Такой формат позволяет проводить детальный анализ производительности на уровне отдельных операций модели.
Несмотря на широкие возможности, TensorFlow Profiler имеет ряд ограничений, существенных в контексте интеграции с MLIR: жесткая привязка к TensorFlow Runtime, отсутствие прямой поддержки промежуточного представления MLIR и сложность извлечения метрик для отдельных операций MLIR. Таким образом, применение данного инструмента ограничено рамками экосистемы TensorFlow.

\subsubsection{PyTorch Profiler}
PyTorch Profiler \cite{pytorch_profiler} предлагает альтернативный подход к профилированию, основанный на использовании Autograd profiler для отслеживания прямого и обратного распространения, а также Kineto backend для низкоуровневого анализа производительности. Интеграция с TensorBoard обеспечивает интерактивную визуализацию результатов профилирования, включая поддержку распределённого профилирования для многоузловых конфигураций.
Однако применимость PyTorch Profiler к задачам, связанным с MLIR, ограничена различиями в архитектуре промежуточного представления и ориентацией на специфичные для PyTorch структуры данных.

\subsubsection{XLA Profiler и связь с MLIR}
Особый интерес представляет XLA (Accelerated Linear Algebra) Profiler, который демонстрирует тесную связь с MLIR в процессе генерации кода. XLA использует MLIR в качестве промежуточного представления, что создаёт основу для интеграции профильных данных. Промежуточное представление XLA HLO (High-Level Optimizer) может быть транслировано в MLIR, обеспечивая прослеживаемость от исходного графа TensorFlow до низкоуровневого кода.
XLA Profiler предоставляет возможности анализа на уровне операций HLO, исследования паттернов fusion и различных оптимизаций. Высокую актуальность данного инструмента для пайплайнов, основанных на MLIR, определяет общность инфраструктуры и возможность переноса подходов профилирования.

\subsubsection{Системные профилировщики}
Системные профилировщики, такие как Intel VTune Profiler, Linux perf и LLVM XRay, обеспечивают микроархитектурный анализ и низкоуровневое профилирование. Intel VTune Profiler позволяет выявлять узкие места (hotspot analysis), анализировать паттерны доступа к памяти и интегрироваться с ML-фреймворками через API. Linux perf использует sampling-based профилирование с генерацией flame graphs для визуализации, однако имеет ограничения при анализе высокоуровневых ML-операций.
LLVM XRay предоставляет функцию трассировки на уровне функций и демонстрирует потенциал для инструментирования кода, сгенерированного из MLIR, что особенно важно для связывания профильных данных с промежуточным представлением.

\subsection{Инструменты анализа и манипуляции MLIR}

\subsubsection{Базовые инструменты MLIR}
MLIR \cite{mlir_main} представляет собой гибридное промежуточное представление, поддерживающее множество требований в унифицированной инфраструктуре. Базовые инструменты MLIR включают \texttt{mlir-opt} для трансформации промежуточного представления с возможностью добавления пользовательских проходов (pass’ов), а также \texttt{mlir-translate} для конвертации между различными форматами, включая импорт TensorFlow SavedModel в MLIR.
\texttt{mlir-opt} обеспечивает расширяемую архитектуру для добавления новых оптимизационных проходов, что создаёт предпосылки для интеграции проходов, работающих с профильными данными. Тем не менее, существующие возможности \texttt{mlir-opt} ограничены в контексте интеграции динамических данных профилирования.

\subsubsection{Визуализация MLIR}
Инструмент \texttt{mlir-to-dot} предназначен для генерации представления MLIR-кода в формате GraphViz, обеспечивая статическую визуализацию структуры промежуточного представления. Основными ограничениями данного инструмента являются отсутствие поддержки метрик времени выполнения и цветового кодирования производительности, что существенно снижает его применимость для анализа производительности.
Альтернативные подходы включают разработку пользовательских проходов для генерации аннотированных графов и интеграцию с внешними инструментами визуализации, что требует значительных дополнительных усилий по разработке.

\subsubsection{Проекты на базе MLIR}
IREE (Intermediate Representation Execution Environment) \cite{iree_main} \cite{iree_tracy} представляет собой MLIR-based end-to-end компилятор и среду выполнения, который компилирует MLIR в исполняемый код со встроенными возможностями профилирования runtime.
IREE обеспечивает инструментирование MLIR-проходов для анализа времени компиляции, что демонстрирует потенциал интеграции профилирования в инфраструктуру MLIR.
Основные ограничения IREE связаны с фокусом на задачах инференса, а не на детальном анализе производительности промежуточного представления.
В сообществе MLIR обсуждается необходимость обеспечения прослеживаемости между компонентами сгенерированного кода и операциями входной спецификации.
XLA-MLIR обеспечивает интеграцию XLA-компилятора \cite{xla_main} с MLIR, создавая предпосылки для переноса возможностей XLA-профилирования в контекст MLIR. ByteIR демонстрирует промышленное применение MLIR с опытом интеграции профилирования и оптимизаций.

\subsection{Метаданные и аннотации в MLIR}

\subsubsection{Системы метаданных MLIR}
MLIR предоставляет развитую систему метаданных, включающую типизированные атрибуты операций, информацию о местоположении (location) для отслеживания источника операций, а также traits и interfaces для определения свойств операций.
 Атрибуты представляют собой статические метаданные, которые могут быть расширены для включения профильных данных, однако их статическая природа ограничивает возможности работы с динамическими runtime-метриками.
Информация о местоположении в MLIR создаёт потенциал для связывания профильных данных с конкретными операциями промежуточного представления.
Traits и interfaces обеспечивают возможности для определения профилируемых операций и расширения функциональности существующих диалектов.

\subsubsection{Опыт LLVM в Profile-Guided Optimization (PGO)}

LLVM предоставляет развитую инфраструктуру для оптимизации на основе профиля (PGO) с workflow \texttt{-fprofile-generate} / \texttt{-fprofile-use} \cite{llvm_pgo} и специализированным форматом профильных данных. Метаданные профиля включают branch weights, block frequencies и интеграцию с оптимизационными проходами.
Применимость LLVM PGO к MLIR ограничена различиями в уровне абстракции и необходимостью адаптации для ML-специфичных метрик, таких как время выполнения тензорных операций и характеристики использования памяти.

\subsubsection{Ограничения текущих подходов}

Анализ существующих подходов выявляет отсутствие стандартизированного формата профильных аннотаций в MLIR, ограниченную поддержку динамических метрик и необходимость ручной интеграции с профилировщиками, что создаёт значительные барьеры для практического применения.
Традиционные методы визуализации производительности включают flame graphs для иерархического представления времени выполнения и heatmaps для цветового кодирования метрик производительности. Flame graphs имеют ограничения при представлении графов операций ML из-за различий в структуре данных, тогда как heatmaps демонстрируют применимость к визуализации вычислительных графов.
TensorBoard Graph Visualization обеспечивает интерактивную визуализацию графов TensorFlow с интеграцией профильных данных через overlays, однако ограничена привязкой к формату TensorFlow. Netron служит универсальным визуализатором нейронных сетей, но не предоставляет интеграции с профильными данными.
Специфические требования к визуализации MLIR включают поддержку иерархической структуры (модули, функции, блоки), цветовое кодирование «горячих» операций, интерактивность для детального анализа метрик и масштабируемость для больших графов.

\subsubsection{Анализ пробелов}
Анализ выявляет критические пробелы в существующей инфраструктуре:
\begin{itemize}
\item Отсутствуют готовые решения для связки профилировщиков ML с промежуточным представлением MLIR.
\item Существующие инструменты не поддерживают цветовое кодирование производительности на уровне операций MLIR.
\item Метаданные MLIR не предназначены для работы с динамическими runtime-метриками.
\item Необходимость использования множества разрозненных инструментов для полного workflow.
\end{itemize}


\subsubsection{Научная и практическая значимость}
Разработка средства автоматического аннотирования MLIR на основе профиля выполнения обладает значимостью в следующих аспектах:
\begin{itemize}
\item Создание моста между профилированием ML-моделей и анализом промежуточного представления MLIR.
\item Предоставление данных для оптимизаций на основе профиля на уровне MLIR.
\item Визуализация производительности для быстрого выявления узких мест.
\end{itemize}

\subsubsection{Ожидаемые результаты}
Реализация предлагаемого средства обеспечит:
\begin{itemize}
\item Повышение эффективности анализа производительности ML-моделей.
\item Создание основы для будущих оптимизаций на основе профильных данных.
\item Расширение инструментария MLIR для практических задач анализа производительности.
\end{itemize}

\newpage