\begin{abstract}

    \begin{center}
        \large{Профилирование и аннотация времени выполнения в MLIR как инструмент для оптимизации программ машинного обучения} \\
    \large\textit{Алексеев Алексей Алексеевич} \\[1 cm]

    Современные модели машинного обучения представляют собой важнейшую область как для фундаментальных исследований, так и для решения прикладных задач в различных областях.
    Одной из актуальных задач в индустрии является создание инфраструктуры, которая позволит эффективно запускать и исполнять эти модели на устройствах с ограниченными ресурсами, на неоднородных и специализированных архитектурах.

    Моя работа направлена на решение проблемы формализации подхода к поиску возможных мест для оптимизаций моделей во время компиляции.
    Этот подход учитывает специфику выполнения программ на различных устройствах.
    Для достижения этой цели используются инструменты многоуровневого промежуточного представления (MLIR), а также динамические профили исполнения программ.

    В рамках работы была разработана система автоматического аннотирования промежуточного представления.
    Эти аннотации могут служить основой для применения дальнейших трансформаций и анализа, что позволит значительно ускорить работу исскустевнного интелекта.

    \vfill
    \end{center}


\end{abstract}
\newpage