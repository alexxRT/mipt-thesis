\begin{abstract}

    \begin{center}
        \large{Профилирование и аннотация времени выполнения в MLIR как инструмент для оптимизации программ машинного обучения} \\
    \large\textit{Алексеев Алексей Алексеевич} \\[1 cm]

    Современные модели машинного обучения представляют большой интерес как для исследовательской направленности, так и для прикладных повседневных задач.
    Одной из текущих задач индустрии является построение инфраструктуры для запуска и исполнения последних на устройствах с ограниченными ресурсами, неоднородной многоядерной архитектурой.

    Множество сложностей возникает на пути адаптации тяжеловесных и требовательных программ машинного обучения. Моя работа призвана решить проблему формализации подхода поиска мест для оптимизаций моделей во время непосредственной компиляции.
    Для этого используются инструменты многоуровневого промежуточного представления и динамические профили исполнения программ.

    В ходе проделанной работы был создан универсальный и расширяемый подход для автоматизированного и наглядного поиска узких мест в архитектуре моделей машинного обучения.
    Подход учитывает особенности выполнения на конкретном устройстве. Также была продемонстрирована работоспособность решения, освещены конкретные места, требующие оптимизаций, и предложены подходы их непосредственной реализации.

    В дальнейшем планируется разработать математические методы оптимизаций на основе полученных результатов. Цель - значительное ускорение исполнения программ машинного обучения.

    \vfill
    \end{center}


\end{abstract}
\newpage