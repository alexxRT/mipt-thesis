\section{Заключение}
\label{sec:Chapter4} \index{Chapter4}

В рамках данной работы было подробно рассмотрено современное решение для разработки компиляторов программ машинного обучения — MLIR.
Благодаря модульной и расширяемой архитектуре удалось интегрировать собственный пайплайн, который добавляет аннотации времени выполнения высокоуровневых операций.
Кроме того, в ходе работы были получены данные о выполнении операций на центральном процессоре с использованием TensorFlow Profiler, которые затем были сохранены в унифицированном формате (DAG).
Преимущество данного подхода было продемонстрировано на примере разработанной утилиты для визуального анализа последовательности произвольных событий, связанных с временем исполнения.
Все разработанные модули были объединены в единый инструмент, который предоставляет возможность автоматического аннотирования высокоуровневых операций собранной информацией о профилировании.
Сравнивая полученные результаты с перечнем поставленных целей, можно утверждать, что все цели работы были успешно достигнуты. Разработанное решение может быть использовано сторонними специалистами на практике для анализа и оптимизации программ машинного обучения с использованием PGO.

\newpage
